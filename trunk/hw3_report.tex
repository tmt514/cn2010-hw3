\documentclass[11pt]{article}
\usepackage{CJK}
\usepackage[margin=2.5cm]{geometry}
%\usepackage{fancyhdr}

\usepackage{palatino}

\pagestyle{empty}
%\lhead{Object Oriented Software Design}
%\chead{Homework 1}
%\rhead{B96201044}
\usepackage{graphicx}

\begin{document}
\begin{CJK}{UTF8}{cwmb}

\title{Computer Network Spring 2010\\Homework \#3 ~ Report}
\author{B96902120 ~施致誠\\B96201044~ 黃上恩}
\date{2010/06/30}
\maketitle

\section{Compilation}
輸入 make 或者\\
\verb|g++ main.cpp serv.cpp mysock.cpp log.cpp dv_algo.cpp -o DV_routing|\\

\section{Distance Vector Algorithm}
每一台機器先初始化所有的距離,以及其 DV table。接下來跑一個 while 迴圈(main.cpp 20-35行),
每一次都先看看有沒有人丟訊息過來或是否有從stdin輸入指令 (呼叫 \verb|Server::Wait()|,裡面呼叫了 \verb|select()|,當 stdin 有東西時會回到 main,否則一直等待封包),如果 \verb|serv_sock| 所使用的 Socket 有任何動靜,就會呼叫 \verb|Server::Refresh()|,更新 Distance Vector。
每次更新 Distance Vector 之後,就會呼叫 \verb|Server::DV_algo()|。在 \verb|Server::DV_algo()|當中,每次都使用 Bellman-Ford Equation 重新計算該 Host 到所有其他 Host 之距離,即針對 $i, j$ 跑兩層迴圈並計算
\[ dis_x[i] = \min_j\{ cost_x[j] + DV[j][i]\} \]
其中 $j$ 跑遍所有 $x$ 的鄰居。如果過程中 $dis_x[i]$ 有任何變更,計算完畢後,會重新呼叫 \verb|Server::Send()| 將更新後的 Distance Vector 發送給所有的鄰居。

\section{Solving Count-to-Infinity Problem}
我們打算使用以下的方法:如果發現目前計算的距離變長了,就一口氣把值改成連往那個點的原本的 cost,再把 DV 傳給 neighbor。這個方法會比原本的快很多,考慮三個點 Server 0, Server 1, Server 2,以及他們之間的距離 $d(0,1)=1, d(0,2)=5, d(1,2)=60$,用原本的方法將 $d(0,2)$ 更新為 100 的時候,對 Server 0 來說需要 28 次的 Refresh,但是使用了新的方法,在 Server 0 上面僅執行了 2 次的 Refresh 。

\end{CJK}
\end{document}